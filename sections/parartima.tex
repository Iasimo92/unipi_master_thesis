
\chapter{Παράρτημα}

\section{Σχεδίαση και Διάγραμμα τοπολογίας δικτύου}

\FloatBarrier

\begin{figure}[h]
	\centering
	\includegraphics[width=1.1\textwidth]{graphics/network_topology_high_level.png}
	\caption{\en{Network topology design}}
\end{figure}

\FloatBarrier

\section{Οδηγός χρήσης της εφαρμογής}

\subsection{Δημιουργία αντικειμένου-Προσθήκη συσκευής}

Προκειμένου να μπορούμε να διαχειριστούμε συσκευή θα πρέπει να την εισάγουμε ως αντικείμενο.

Ανοίγουμε το \en{http://127.0.0.1:8000/admin/} σε \en{browser} και εισάγουμε ως χρήστη \en{iasonas}
και κωδικό \en{ericsson}.

\FloatBarrier

\begin{figure}[h]
	\centering
	\includegraphics[width=1.1\textwidth]{graphics/GUI_LOGIN.png}
	\caption{\en{GUI Login}}
\end{figure}

\FloatBarrier

Αφού συνδεθούμε στο \en{GUI} επιλέγουμε \en{Network}, \en{Devices} και \en{Add} προκειμένου να εισάγουμε καινούργια συσκευή.

\FloatBarrier

\begin{figure}[h]
	\centering
	\includegraphics[width=1.1\textwidth]{graphics/DJANGO_ADMIN.png}
	\caption{\en{GUI Login second page}}
\end{figure}

\FloatBarrier

Στη συνέχεια μας εμφανίζεται η παρακάτω εικόνα. Βάζουμε τα στοιχεία της συσκευής και πατάμε αποθήκευση.

\FloatBarrier

\begin{figure}[h]
	\centering
	\includegraphics[width=1.1\textwidth]{graphics/ADD_DEVICE.png}
	\caption{\en{Add device page}}
\end{figure}

\FloatBarrier

Αφού το κάνουμε αυτό όλες οι συσκευές που προσθέσαμε μπορούμε να τις δούμε σε όλες τις λειτουργίες της εφαρμογής.
Οι λειτουργίες μπορούν να εκτελεστούν μια μια όπως έγινε στο \en{application demo}.

\section{Κώδικας}

H εφαρμογή είναι ελεύθερη στη σελίδα: \selectlanguage{english} \url{https://github.com/Iasimo92/SDN_Django_framework_for_implementation_network_service_configuration_application}. \selectlanguage{greek}


\FloatBarrier

\begin{figure}[h]
	\centering
	\includegraphics[width=1.1\textwidth]{graphics/github_page.png}
	\caption{\en{github repo}}
\end{figure}

\FloatBarrier

