\chapter{Εισαγωγή}

%Το \en{Lorem Ipsum} είναι απλά ένα κείμενο χωρίς νόημα για τους επαγγελματίες της τυπογραφίας και στοιχειοθεσίας \cite{LoremIpsumAll}. Το \en{Lorem Ipsum} είναι το επαγγελματικό πρότυπο όσον αφορά το κείμενο χωρίς νόημα, από τον 15ο αιώνα, όταν ένας ανώνυμος τυπογράφος πήρε ένα δοκίμιο και ανακάτεψε τις λέξεις για να δημιουργήσει ένα δείγμα βιβλίου. Όχι μόνο επιβίωσε πέντε αιώνες, αλλά κυριάρχησε στην ηλεκτρονική στοιχειοθεσία, παραμένοντας με κάθε τρόπο αναλλοίωτο. Έγινε δημοφιλές τη δεκαετία του '60 με την έκδοση των δειγμάτων της \en{Letraset} όπου περιελάμβαναν αποσπάσματα του \en{Lorem Ipsum}, και πιο πρόσφατα με το λογισμικό ηλεκτρονικής σελιδοποίησης όπως το \en{Aldus PageMaker} που περιείχαν εκδοχές του \en{Lorem Ipsum}.
\section{Στόχοι του έργου}

Ο βασικός στόχος αυτής της διπλωματικής εργασίας είναι η 
ανάπτυξη μιας ολοκληρωμένης εφαρμογής που θα ενσωματώνει τις δυνατότητες της αυτοματοποίησης δικτύων, της 
ανάπτυξης εφαρμογών Ιστού, και των τεχνολογιών οριζόμενων από λογισμικό δικτύων (\en{SDN}). 
Μέσα από την πρακτική εφαρμογή, επιχειρείται η κατανόηση και αξιολόγηση της λειτουργικότητας εργαλείων όπως το 
\en{Kubernetes}, το \en{Django}, και η γλώσσα προγραμματισμού \en{Python}. Παράλληλα, διερευνώνται οι πρακτικές δυνατότητες της αυτοματοποίησης 
στη διαχείριση δικτύων, την παρακολούθηση και τη συντήρηση, καθώς και οι τρόποι με τους οποίους αυτές μπορούν να βελτιώσουν τη λειτουργικότητα 
και την αποτελεσματικότητα.

\section{Περιγραφή προβλήματος και λύσης}
Η διαχείριση δικτυακών υποδομών έχει γίνει εξαιρετικά περίπλοκη λόγω του μεγέθους και της πολυπλοκότητας των σύγχρονων δικτύων. 
Η χειροκίνητη διαχείριση αυτών των υποδομών είναι χρονοβόρα και επιρρεπής σε σφάλματα, ενώ δεν μπορεί να ανταποκριθεί επαρκώς στις 
αυξανόμενες απαιτήσεις για ευελιξία, ταχύτητα και αξιοπιστία. Τα παραδοσιακά μοντέλα διαχείρισης συσκευών απαιτούν εξειδικευμένες γνώσεις, 
καθιστώντας δύσκολη την προσαρμογή στις ταχέως μεταβαλλόμενες συνθήκες.

Η λύση που προτείνεται στην παρούσα εργασία περιλαμβάνει την υλοποίηση μιας εφαρμογής που αξιοποιεί τεχνολογίες αυτοματοποίησης 
για να μειώσει την ανθρώπινη παρέμβαση, να εξαλείψει επαναλαμβανόμενες εργασίες και να ενισχύσει τη δυνατότητα λήψης αποφάσεων σε πραγματικό χρόνο. 
Μέσα από τη χρήση βιβλιοθηκών \en{Python}, της πλατφόρμας \en{Django}, και της τεχνολογίας \en{Kubernetes}, επιτυγχάνεται η κεντρικοποιημένη 
διαχείριση και η δυναμική προσαρμογή της εφαρμογής σύμφωνα με τις ανάγκες του οργανισμού.

\section{Τεχνολογίες που χρησιμοποιήθηκαν και γιατί}



Η παρούσα εργασία επικεντρώνεται στην ανάπτυξη μιας εφαρμογής 
για τη διαχείριση δικτυακών υποδομών, χρησιμοποιώντας σύγχρονες 
τεχνολογίες και εργαλεία. Η \en{Python} αποτέλεσε τη βασική γλώσσα 
προγραμματισμού, χάρη στην ευκολία της στη σύνταξη κώδικα, αλλά και 
στη μεγάλη συλλογή βιβλιοθηκών που προσφέρει, ειδικά για δικτυακές 
εφαρμογές και αυτοματοποίηση. Μέσω της ενσωμάτωσης πρωτοκόλλων όπως 
το \en{SSH} και το \en{REST}, καταφέραμε να επιτύχουμε την 
αποτελεσματική διαχείριση συσκευών και τη διεκπεραίωση κρίσιμων 
λειτουργιών. Η \en{Python} χρησιμοποιήθηκε κυρίως για την 
αυτοματοποίηση διαδικασιών, την ανάπτυξη \en{scripts} που 
παρακολουθούν τη λειτουργία των συσκευών, καθώς και για την 
υλοποίηση \en{API} που ενισχύουν τη διαλειτουργικότητα της εφαρμογής.

Για την ανάπτυξη του \en{backend}, επιλέξαμε το \en{Django}, 
το οποίο ξεχωρίζει για την ευελιξία, την ασφάλεια και την 
ταχύτητα ανάπτυξης που προσφέρει. Το \en{Django} 
αξιοποιήθηκε για τη δημιουργία της κεντρικής διεπαφής διαχείρισης 
των δικτύων, επιτρέποντας την επεξεργασία δεδομένων και 
την παροχή δυναμικών υπηρεσιών στους χρήστες. Με τη χρήση αυτού 
του ισχυρού πλαισίου, μπορέσαμε να διαχειριστούμε δεδομένα με 
τρόπο ασφαλή και αξιόπιστο, ενώ παράλληλα υποστηρίξαμε την ταχύτερη 
ανάπτυξη της εφαρμογής.

Για τη διαχείριση των \en{microservices} και την εξασφάλιση της 
κλιμακωσιμότητας της εφαρμογής, υιοθετήσαμε το \en{Kubernetes}. 
Η χρήση του \en{Kubernetes} μας έδωσε τη δυνατότητα να 
διαχειριστούμε κοντέινερ που φιλοξενούσαν τα \en{microservices}, 
επιτρέποντας τη δυναμική ανάπτυξη, την αποτελεσματική κατανομή πόρων 
και τη συντήρηση της εφαρμογής. Αυτή η προσέγγιση εξασφάλισε ότι η 
εφαρμογή θα μπορούσε να προσαρμοστεί σε αυξημένες απαιτήσεις φορτίου, 
ενώ παράλληλα μειώθηκε ο χρόνος διαχείρισης και η πολυπλοκότητα των 
υποδομών κάτι το οποίο θα είχε μεγαλύτερη χρησιμότητα σε περιβάλλοντα παραγωγής.

Ένα ιδιαίτερα σημαντικό στοιχείο της εργασίας ήταν η δυνατότητα 
δοκιμής της εφαρμογής σε περιβάλλοντα που προσομοιώνουν 
πραγματικές συνθήκες. Για τον σκοπό αυτό, δημιουργήσαμε εικονικά 
δίκτυα χρησιμοποιώντας το εργαλείο \en{GNS3} (\en{Graphical Network Simulator-3}), 
σε συνδυασμό με \en{Cisco Images} και \en{VirtualBox}. 
Αυτή η υποδομή επέτρεψε την εξομοίωση πραγματικών δικτυακών 
περιβαλλόντων, διευκολύνοντας την ανίχνευση και την επίλυση 
προβλημάτων πριν από την εφαρμογή του συστήματος σε πραγματικά δίκτυα. 
Η διαδικασία αυτή αποδείχθηκε καθοριστική, καθώς μας επέτρεψε να βελτιώσουμε την αξιοπιστία και τη λειτουργικότητα της εφαρμογής, μειώνοντας σημαντικά τον κίνδυνο αποτυχίας σε πραγματικές συνθήκες.

Η εργασία οργανώθηκε σε κεφάλαια που καλύπτουν κάθε πτυχή της 
υλοποίησης. Στο πρώτο κεφάλαιο παρουσιάζεται μια εισαγωγή 
στο θέμα της πτυχιακής, όπου αναλύονται οι στόχοι και το γενικό 
πλαίσιο της εργασίας. Στο δεύτερο κεφάλαιο περιγράφονται οι 
απαιτήσεις και οι προδιαγραφές του έργου, με αναφορά στις 
τεχνολογίες και τα εργαλεία που χρησιμοποιήθηκαν.

Στη συνέχεια, στο τρίτο κεφάλαιο, αναλύεται η 
μεθοδολογία ανάπτυξης της εφαρμογής, ενώ το τέταρτο κεφάλαιο εστιάζει 
στην υλοποίηση της εφαρμογής, περιγράφοντας βήμα προς βήμα την 
αρχιτεκτονική και τις επιμέρους λειτουργίες της. Στο πέμπτο κεφάλαιο 
γίνεται εκτενής αναφορά στις σύγχρονες τεχνολογίες \en{containerization}, 
εξηγώντας τα οφέλη και τη συνεισφορά τους στην ανάπτυξη της εφαρμογής.

Στο έκτο κεφάλαιο πραγματοποιείται αξιολόγηση της εφαρμογής, 
βασισμένη σε δοκιμές που έγιναν σε εικονικά περιβάλλοντα. 
Αυτές οι δοκιμές μας επέτρεψαν να αξιολογήσουμε τη σταθερότητα και 
την αποτελεσματικότητα του συστήματος. Στο έβδομο κεφάλαιο παρουσιάζονται 
τα συμπεράσματα της εργασίας και συζητώνται οι δυνατότητες μελλοντικής 
επέκτασης της εφαρμογής. Τέλος, στο όγδοο κεφάλαιο περιλαμβάνεται ένα 
Παράρτημα με χρήσιμες λεπτομέρειες για την εργασία, ενώ στο ένατο 
κεφάλαιο παρατίθεται η Βιβλιογραφία.

Η εργασία αυτή συνδυάζει θεωρητικές γνώσεις με πρακτική εφαρμογή, 
αξιοποιώντας σύγχρονες τεχνολογίες για τη δημιουργία μιας καινοτόμου 
και αξιόπιστης λύσης στη διαχείριση δικτύων. 