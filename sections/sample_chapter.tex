\chapter{Εισαγωγή}

%Το \en{Lorem Ipsum} είναι απλά ένα κείμενο χωρίς νόημα για τους επαγγελματίες της τυπογραφίας και στοιχειοθεσίας \cite{LoremIpsumAll}. Το \en{Lorem Ipsum} είναι το επαγγελματικό πρότυπο όσον αφορά το κείμενο χωρίς νόημα, από τον 15ο αιώνα, όταν ένας ανώνυμος τυπογράφος πήρε ένα δοκίμιο και ανακάτεψε τις λέξεις για να δημιουργήσει ένα δείγμα βιβλίου. Όχι μόνο επιβίωσε πέντε αιώνες, αλλά κυριάρχησε στην ηλεκτρονική στοιχειοθεσία, παραμένοντας με κάθε τρόπο αναλλοίωτο. Έγινε δημοφιλές τη δεκαετία του '60 με την έκδοση των δειγμάτων της \en{Letraset} όπου περιελάμβαναν αποσπάσματα του \en{Lorem Ipsum}, και πιο πρόσφατα με το λογισμικό ηλεκτρονικής σελιδοποίησης όπως το \en{Aldus PageMaker} που περιείχαν εκδοχές του \en{Lorem Ipsum}.
\section{Στόχοι του έργου}

Ο βασικός στόχος αυτής της διπλωματικής εργασίας είναι η 
ανάπτυξη μιας ολοκληρωμένης εφαρμογής που θα ενσωματώνει τις δυνατότητες της αυτοματοποίησης δικτύων, της 
ανάπτυξης εφαρμογών Ιστού, και των τεχνολογιών οριζόμενων από λογισμικό δικτύων (\en{SDN}). 
Μέσα από την πρακτική εφαρμογή, επιχειρείται η κατανόηση και αξιολόγηση της λειτουργικότητας εργαλείων όπως το 
\en{Kubernetes}, το \en{Django}, και η γλώσσα προγραμματισμού \en{Python}. Παράλληλα, διερευνώνται οι πρακτικές δυνατότητες της αυτοματοποίησης 
στη διαχείριση δικτύων, την παρακολούθηση και τη συντήρηση, καθώς και οι τρόποι με τους οποίους αυτές μπορούν να βελτιώσουν τη λειτουργικότητα 
και την αποτελεσματικότητα.

\section{Περιγραφή προβλήματος και λύσης}
Η διαχείριση δικτυακών υποδομών έχει γίνει εξαιρετικά περίπλοκη λόγω του μεγέθους και της πολυπλοκότητας των σύγχρονων δικτύων. 
Η χειροκίνητη διαχείριση αυτών των υποδομών είναι χρονοβόρα και επιρρεπής σε σφάλματα, ενώ δεν μπορεί να ανταποκριθεί επαρκώς στις 
αυξανόμενες απαιτήσεις για ευελιξία, ταχύτητα και αξιοπιστία. Τα παραδοσιακά μοντέλα διαχείρισης συσκευών απαιτούν εξειδικευμένες γνώσεις, 
καθιστώντας δύσκολη την προσαρμογή στις ταχέως μεταβαλλόμενες συνθήκες.

Η λύση που προτείνεται στην παρούσα εργασία περιλαμβάνει την υλοποίηση μιας εφαρμογής που αξιοποιεί τεχνολογίες αυτοματοποίησης 
για να μειώσει την ανθρώπινη παρέμβαση, να εξαλείψει επαναλαμβανόμενες εργασίες και να ενισχύσει τη δυνατότητα λήψης αποφάσεων σε πραγματικό χρόνο. 
Μέσα από τη χρήση βιβλιοθηκών \en{Python}, της πλατφόρμας \en{Django}, και της τεχνολογίας \en{Kubernetes}, επιτυγχάνεται η κεντρικοποιημένη 
διαχείριση και η δυναμική προσαρμογή των δικτύων σύμφωνα με τις ανάγκες του οργανισμού.

\section{Τεχνολογίες που χρησιμοποιήθηκαν και γιατί}


\textbf{\en{Python}}:

Λόγοι επιλογής: 

Η ευκολία στη σύνταξη και η πλούσια συλλογή βιβλιοθηκών για δικτυακές εφαρμογές και αυτοματοποίηση καθιστούν τη \en{Python} ιδανική για την ανάπτυξη της εφαρμογής. Επιπλέον, η ενσωμάτωσή της με πρωτόκολλα όπως \en{SSH} και \en{REST} επιτρέπει την αποτελεσματική διαχείριση συσκευών.

Χρήση: 

Αυτοματοποιημένες διαδικασίες, ανάπτυξη \en{scripts} για παρακολούθηση συσκευών και υλοποίηση \en{API}.

\textbf{\en{Django}}:

Λόγοι επιλογής: 

Το ισχυρό \en{backend} περιβάλλον του \en{Django} παρέχει ευελιξία, ασφάλεια, και δυνατότητα γρήγορης ανάπτυξης εφαρμογών Ιστού.

Χρήση: 

Δημιουργία της κεντρικής διεπαφής διαχείρισης δικτύων, επεξεργασία δεδομένων και παροχή δυναμικών υπηρεσιών στον χρήστη.

\textbf{\en{Kubernetes}}:

Λόγοι επιλογής: 

Η δυνατότητα κλιμάκωσης και διαχείρισης \en{microservices} μέσω \en{containers} διευκολύνει τη δυναμική ανάπτυξη και συντήρηση της εφαρμογής.

Χρήση: 

Ανάπτυξη και κλιμάκωση \en{microservices} που υποστηρίζουν τις λειτουργίες του συστήματος.

\textbf{Περιβάλλον ανάπτυξης (\en{GNS3 VM}, \en{Cisco Images}, \en{VirtualBox})}:

Λόγοι επιλογής: 

Εξομοίωση πραγματικών δικτυακών περιβαλλόντων για δοκιμή και αξιολόγηση της εφαρμογής.

Χρήση: 

Δοκιμές αυτοματοποίησης και προσομοίωση διαμόρφωσης δικτύων.


Η δομή της εργασίας έχει ως εξής:
\begin{itemize}
    \item Στο Κεφάλαιο 1, γίνεται μια εισαγωγή στο θέμα της πτυχιακής και παρουσιάζονται οι στόχοι της εργασίας.
    \item Στο Κεφάλαιο 2, αναλύονται οι απαιτήσεις και οι προδιαγραφές του έργου, περιλαμβάνοντας τις τεχνολογίες και τα εργαλεία που χρησιμοποιήθηκαν.
    \item Στο Κεφάλαιο 3, περιγράφεται η μεθοδολογία ανάπτυξης της εφαρμογής.
    \item Στο Κεφάλαιο 4, παρουσιάζονται η υλοιποίηση της εφαρμογής.
    \item Στο Κεφάλαιο 5, γίνεται περιγραφή των νέων τεχνολογιών(\en{containerization})
    \item Στο Κεφάλαιο 6, γίνονται δοκιμές και αξιολόγηση της εφαρμογής
    \item Στο Κεφάλαιο 7, παρουσιάζεται το συμπέρασμα και μελλοντικές επεκτάσεις της εργασίας αυτής.
    \item Στο Κεφάλαιο 8, περιέχεται ένα Παράρτημα και στο Κεφάλαιο 9 η Βιβλιογραφία.
\end{itemize}