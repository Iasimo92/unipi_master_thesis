\chapter{Εισαγωγή}

%Το \en{Lorem Ipsum} είναι απλά ένα κείμενο χωρίς νόημα για τους επαγγελματίες της τυπογραφίας και στοιχειοθεσίας \cite{LoremIpsumAll}. Το \en{Lorem Ipsum} είναι το επαγγελματικό πρότυπο όσον αφορά το κείμενο χωρίς νόημα, από τον 15ο αιώνα, όταν ένας ανώνυμος τυπογράφος πήρε ένα δοκίμιο και ανακάτεψε τις λέξεις για να δημιουργήσει ένα δείγμα βιβλίου. Όχι μόνο επιβίωσε πέντε αιώνες, αλλά κυριάρχησε στην ηλεκτρονική στοιχειοθεσία, παραμένοντας με κάθε τρόπο αναλλοίωτο. Έγινε δημοφιλές τη δεκαετία του '60 με την έκδοση των δειγμάτων της \en{Letraset} όπου περιελάμβαναν αποσπάσματα του \en{Lorem Ipsum}, και πιο πρόσφατα με το λογισμικό ηλεκτρονικής σελιδοποίησης όπως το \en{Aldus PageMaker} που περιείχαν εκδοχές του \en{Lorem Ipsum}.
\section{Στόχοι του έργου}

Ο πρωταρχικός σκοπός της παρούσας διπλωματικής εργασίας είναι η ανάπτυξη μιας ολοκληρωμένης εφαρμογής που ενσωματώνει τις δυνατότητες της αυτοματοποίησης δικτύων 
και της ανάπτυξης εφαρμογών Ιστού. Η μελέτη επικεντρώνεται στη χρήση σύγχρονων τεχνολογιών, συμπεριλαμβανομένων του \en{Kubernetes}, του \en{Django} και της γλώσσας προγραμματισμού \en{Python}.
Ο βασικός στόχος αυτής της διπλωματικής εργασίας είναι η ανάπτυξη μιας ολοκληρωμένης εφαρμογής που ενσωματώνει τις δυνατότητες της αυτοματοποίησης δικτύων και της ανάπτυξης εφαρμογών Ιστού. Η εργασία εστιάζει στη χρήση σύγχρονων εργαλείων όπως το \en{Kubernetes}, το \en{Django} και την γλώσσα προγραμματισμού \en{Python}.   

Συγκεκριμένα, η εφαρμογή που αναπτύχθηκε επιτρέπει την αυτοματοποιημένη διαχείριση δικτυακών συσκευών μέσω ενός γραφικού περιβάλλοντος. Η υλοποίηση βασίστηκε στο \en{Django}, 
το οποίο χρησιμοποιήθηκε ως το βασικό \en{backend framework}. Χάρη στο \en{Django}, δημιουργήθηκαν οι απαραίτητες συναρτήσεις και ένα απλό \en{frontend} περιβάλλον, επιτρέποντας την εστίαση στη λειτουργικότητα της εφαρμογής, αντί της εμφάνισης προς τον χρήστη.
Για τη δοκιμή και αξιολόγηση της εφαρμογής, αξιοποιήθηκε το \en{GNS3}, ένα εργαλείο προσομοίωσης δικτυακών συσκευών. Το \en{GNS3} επέτρεψε την προσομοίωση ενός πραγματικού δικτυακού περιβάλλοντος, διευκολύνοντας τις δοκιμές και την επικοινωνία του \en{Django service} με εικονικές συσκευές. Χωρίς το \en{GNS3}, η διαδικασία αυτή θα απαιτούσε φυσικό εξοπλισμό, γεγονός που θα καθιστούσε τις δοκιμές ιδιαίτερα απαιτητικές τόσο σε κόστος όσο και σε πολυπλοκότητα, λόγω της ανάγκης για αγορά και της δικτυακή ενσωμάτωσή του.

Όταν η εφαρμογή έφτασε σε ένα ικανοποιητικό επίπεδο ανάπτυξης, ενσωματώθηκαν \en{Cloud Native} τεχνολογίες, προκειμένου να ακολουθήσει τις σύγχρονες τάσεις στον χώρο της Πληροφορικής και των δικτύων. 
Για την υλοποίηση αυτής της προσέγγισης, δημιουργήθηκε σε ένα \en{Linux} περιβάλλον το κατάλληλο περιβάλλον εκτέλεσης, ώστε η εφαρμογή να μετατραπεί σε \en{Image} και μετά \en{container} μέσα σε ένα \en{Kubernetes pod}. Στη συνέχεια, η διαχείρισή της έγινε μέσω του \en{Kubernetes}, το οποίο επιτρέπει την εύκολη ανάπτυξη, συντήρηση και κλιμάκωση των υπηρεσιών.


\section{Περιγραφή προβλήματος και λύσης}
Η διαχείριση δικτυακών υποδομών έχει γίνει εξαιρετικά περίπλοκη λόγω του μεγέθους και της πολυπλοκότητας των σύγχρονων δικτύων. 
Η χειροκίνητη διαχείριση αυτών των υποδομών είναι χρονοβόρα και επιρρεπής σε σφάλματα διαδικασία, ενώ δεν μπορεί να ανταποκριθεί επαρκώς στις 
αυξανόμενες απαιτήσεις για ευελιξία, ταχύτητα και αξιοπιστία. Τα παραδοσιακά μοντέλα διαχείρισης συσκευών απαιτούν εξειδικευμένες γνώσεις, 
καθιστώντας δύσκολη την προσαρμογή στις ταχέως μεταβαλλόμενες συνθήκες. Όταν μιλάμε για τα παραδοσιακά μοντέλα διαχείρισης συσκευών, αναφερόμαστε στη χειροκίνητη διαδικασία μέσω της οποίας οι μηχανικοί πραγματοποιούσαν αλλαγές και ρυθμίσεις στις υποδομές. Αυτή η προσέγγιση απαιτούσε άμεση ανθρώπινη παρέμβαση, καθιστώντας τη διαδικασία πιο χρονοβόρα, επιρρεπή σε σφάλματα και δύσκολα διαχειρίσιμη σε μεγάλης κλίμακας περιβάλλοντα. 

Αν και η εφαρμογή μας δεν περιλαμβάνει ιδιαίτερα σύνθετους μηχανισμούς αυτοματοποιημένης διαχείρισης, στις μεγάλες επιχειρήσεις 
σύγχρονα εργαλεία αυτού του τύπου χρησιμοποιούνται για την υλοποίηση πιο περίπλοκων διαδικασιών. Παρόλα αυτά, η προτεινόμενη λύση συμβάλλει στην αντιμετώπιση παρόμοιων προκλήσεων, μειώνοντας σημαντικά την πιθανότητα ανθρώπινου σφάλματος και βελτιώνοντας την αποδοτικότητα των δικτυακών διαχειριστικών εργασιών


Η λύση που προτείνεται στην παρούσα εργασία περιλαμβάνει την υλοποίηση μιας εφαρμογής που αξιοποιεί ένα \en{Django backend} για την παραμετροποίηση των δικτυακών συσκευών με στόχο να μειώσει την ανθρώπινη παρέμβαση, να εξαλείψει επαναλαμβανόμενες εργασίες και να ενισχύσει τη δυνατότητα λήψης αποφάσεων σε πραγματικό χρόνο. Η διαχείριση δικτυακών συσκευών απαιτεί εξειδικευμένες γνώσεις και εξοικείωση με το περιβάλλον γραμμής εντολών (\en{Command Line Interface - CLI}). Ωστόσο, με τη χρήση μιας εφαρμογής που αυτοματοποιεί ή απλοποιεί τις διαδικασίες διαχείρισης, η ανάγκη για αυτή την εξειδίκευση περιορίζεται σημαντικά, καθώς απαιτείται μόνο η κατανόηση θεμελιωδών δικτυακών εννοιών. Συνεπώς, η ανάγκη εκπαίδευσης του ανθρώπινου δυναμικού σε εξειδικευμένα αντικείμενα μειώνεται, καθιστώντας τη διαδικασία εκμάθησης λιγότερο χρονοβόρα και απαιτητική.

Μέσα από τη χρήση διαφόρων βιβλιοθηκών \en{Python}, των εργαλείων και βιβλιοθηκών \en{Django}, και της τεχνολογίας \en{Kubernetes}, επιτυγχάνεται η κεντρικοποιηµένη διαχείριση και η δυναµική προσαρµογή της εφαρμογής σύµφωνα µε τις ανάγκες του οργανισµού. Η δυνατότητα δυναμικής προσαρμογής της εφαρμογής δεν υλοποιήθηκε στο πλαίσιο της παρούσας διπλωματικής εργασίας, ωστόσο αναφέρεται ως μια προοπτική που θα μπορούσε να αξιοποιηθεί αν αυτή η εφαρμογή υλοποιούνταν σε ένα πραγματικό και μεγαλύτερο περιβάλλον. Σε περιπτώσεις αυξημένης ζήτησης για υπηρεσίες, η ανάγκη για επέκταση της εφαρμογής θα μπορούσε να καλυφθεί αυτόματα, εξασφαλίζοντας αποδοτικότητα και ευελιξία. Στο \en{Kubernetes}, αυτό επιτυγχάνεται μέσω του \en{Horizontal Pod Autoscaling}, το οποίο επιτρέπει την προσαρμογή της εφαρμογής ανάλογα με τους διαθέσιμους πόρους και τις απαιτήσεις για ζήτηση υπηρεσίας.  

\section{Τεχνολογίες που χρησιμοποιήθηκαν και γιατί}



Η παρούσα εργασία επικεντρώνεται στην ανάπτυξη μιας εφαρμογής 
για τη διαχείριση δικτυακών υποδομών, χρησιμοποιώντας σύγχρονες 
τεχνολογίες και εργαλεία. Η \en{Python} αποτέλεσε τη βασική γλώσσα 
προγραμματισμού, χάρη στην ευκολία της στη σύνταξη κώδικα, αλλά και 
στη μεγάλη συλλογή βιβλιοθηκών που προσφέρει, ειδικά για δικτυακές 
εφαρμογές και αυτοματοποίηση. Μέσω της ενσωμάτωσης πρωτοκόλλων όπως 
το \en{SSH} και το \en{REST}, καταφέραμε να επιτύχουμε την 
αποτελεσματική διαχείριση συσκευών και τη διεκπεραίωση κρίσιμων 
λειτουργιών. Η \en{Python} χρησιμοποιήθηκε κυρίως για την 
αυτοματοποίηση διαδικασιών, την ανάπτυξη \en{scripts} που 
παρακολουθούν τη λειτουργία των συσκευών, καθώς και για την 
υλοποίηση \en{API} που ενισχύουν τη διαλειτουργικότητα της εφαρμογής.

Για την ανάπτυξη του \en{backend}, επιλέξαμε το \en{Django}, 
το οποίο ξεχωρίζει για την ευελιξία, την ασφάλεια και την 
ταχύτητα ανάπτυξης που προσφέρει. Το \en{Django} 
αξιοποιήθηκε για τη δημιουργία της κεντρικής διεπαφής διαχείρισης 
των δικτύων, επιτρέποντας την επεξεργασία δεδομένων και 
την παροχή δυναμικών υπηρεσιών στους χρήστες. Με τη χρήση αυτού 
του ισχυρού πλαισίου, μπορέσαμε να διαχειριστούμε δεδομένα με 
τρόπο ασφαλή και αξιόπιστο, ενώ παράλληλα υποστηρίξαμε την ταχύτερη 
ανάπτυξη της εφαρμογής.

Για τη διαχείριση των \en{microservices} και την εξασφάλιση της 
κλιμακωσιμότητας της εφαρμογής, υιοθετήσαμε το \en{Kubernetes}. 
Η χρήση του \en{Kubernetes} μας έδωσε τη δυνατότητα να 
διαχειριστούμε κοντέινερ που φιλοξενούσαν τα \en{microservices}, 
επιτρέποντας τη δυναμική ανάπτυξη, την αποτελεσματική κατανομή πόρων 
και τη συντήρηση της εφαρμογής. Αυτή η προσέγγιση εξασφάλισε ότι η 
εφαρμογή θα μπορούσε να προσαρμοστεί σε αυξημένες απαιτήσεις φορτίου, 
και αυξημένη ζήτηση για υπηρεσία.

Ένα ιδιαίτερα σημαντικό στοιχείο της εργασίας ήταν η δυνατότητα 
δοκιμής της εφαρμογής σε περιβάλλοντα που προσομοιώνουν 
πραγματικές συνθήκες. Για τον σκοπό αυτό, δημιουργήσαμε εικονικά 
δίκτυα χρησιμοποιώντας το εργαλείο \en{GNS3} (\en{Graphical Network Simulator-3}), 
σε συνδυασμό με \en{Cisco Images} και \en{VirtualBox}. 
Αυτή η υποδομή επέτρεψε την εξομοίωση πραγματικών δικτυακών 
συσκευών, διευκολύνοντας την ανίχνευση και την επίλυση 
προβλημάτων πριν από την εφαρμογή του συστήματος σε πραγματικά δίκτυα. 
Η διαδικασία αυτή αποδείχθηκε καθοριστική, καθώς μας επέτρεψε να βελτιώσουμε την αξιοπιστία και τη λειτουργικότητα της εφαρμογής, μειώνοντας σημαντικά τον κίνδυνο αποτυχίας σε πραγματικές συνθήκες.


\section{Διάρθρωση της παρούσας εργασίας}

Η παρούσα πτυχιακή εργασία αποτελεί το αποτέλεσμα μιας συστηματικής και επίμονης προσπάθειας, με στόχο την ανάπτυξη μιας ολοκληρωμένης εφαρμογής, αξιοποιώντας σύγχρονες τεχνολογίες ανάπτυξης λογισμικού. Μέσα από αυτήν την εργασία, επιχειρείται η αναλυτική παρουσίαση όλων των σταδίων της διαδικασίας ανάπτυξης, από τη θεωρητική θεμελίωση έως την τελική υλοποίηση και αξιολόγηση.

Στο πρώτο κεφάλαιο γίνεται ένας πρόλογος της εργασίας ενώ στο τρίτο παρουσιάζεται η εισαγωγή στο θέμα της πτυχιακής εργασίας. Αναλύονται οι στόχοι, το γενικό πλαίσιο και η σπουδαιότητα της εφαρμογής. Γίνεται μια σύντομη αναφορά στο πρόβλημα που επιχειρείται να επιλυθεί, καθώς και στη συνεισφορά της εργασίας στην ευρύτερη επιστημονική κοινότητα.

Το τέταρτο κεφάλαιο επικεντρώνεται στο θεωρητικό υπόβαθρο, παρουσιάζοντας τις βασικές αρχές, τα μοντέλα και τις τεχνολογίες που υποστηρίζουν την ανάπτυξη της εφαρμογής. Αναλύονται οι απαιτήσεις του έργου και περιγράφονται τα εργαλεία που χρησιμοποιήθηκαν.

Στο πέμπτο κεφάλαιο περιγράφεται η σχεδίαση της εφαρμογής. Παρουσιάζεται η μεθοδολογία ανάπτυξης, οι αρχιτεκτονικές επιλογές και οι σχεδιαστικές αποφάσεις που ελήφθησαν. Επιπλέον, αναλύεται η λογική δομή της εφαρμογής μέσω διαγραμμάτων και περιγράφονται οι λειτουργικές και μη λειτουργικές απαιτήσεις.

Το έκτο κεφάλαιο καλύπτει την υλοποίηση της εφαρμογής. Περιγράφονται βήμα προς βήμα τα επιμέρους στάδια ανάπτυξης, οι τεχνολογίες που χρησιμοποιήθηκαν, καθώς και οι προκλήσεις που προέκυψαν. Παρουσιάζονται, επίσης, τα κύρια χαρακτηριστικά της εφαρμογής και οι λειτουργίες που υλοποιήθηκαν ενώ γίνεται και μια εκτεταμένη παρουσίαση του \en{testbed}.

Στο έβδομο κεφάλαιο γίνεται επίδειξη της εφαρμογής. Παρουσιάζεται η λειτουργία της εφαρμογής μέσα από παραδείγματα χρήσης, καθώς και τα αποτελέσματα που επιτεύχθηκαν. 

Το όγδοο κεφάλαιο εστιάζει στη διαδικασία \en{containerization} και στην ανάπτυξη της εφαρμογής σε περιβάλλοντα κυβερνήτη. Εξηγείται η επιλογή της συγκεκριμένης τεχνολογίας, τα πλεονεκτήματά της και τα βήματα που ακολουθήθηκαν για την ολοκλήρωση της διαδικασίας μετατροπής \en{Image},\en{container} μέχρι και την υλοποίηση κυβερνήτη.

Στο ένατο κεφάλαιο παρουσιάζονται τα συμπεράσματα που προέκυψαν από την εκπόνηση της εργασίας. Αξιολογείται η απόδοση της εφαρμογής, καταγράφονται τα διδάγματα που αντλήθηκαν από τη διαδικασία ανάπτυξης, ενώ παράλληλα προτείνονται ιδέες για μελλοντικές βελτιώσεις και επεκτάσεις.

Η εργασία αυτή στοχεύει να προσφέρει μια ολοκληρωμένη εικόνα της ανάπτυξης μιας σύγχρονης εφαρμογής, συνδυάζοντας θεωρία, σχεδιασμό, υλοποίηση και τεχνολογίες αιχμής, όπως το \en{containerization}, σε ένα ενιαίο, δομημένο πλαίσιο.



