
\chapter{Συμπέρασμα}

\section{Εισαγωγή}

Μετά από δυόμιση μήνες εντατικής δουλειάς, μελέτης και 
προγραμματισμού για αυτή τη διπλωματική εργασία, είμαι πλέον 
σίγουρος ότι έχω επιτύχει τους περισσότερους από τους στόχους που 
έθεσα στην αρχή.

Επιπλέον, η αναζήτηση της κατάλληλης έκδοσης \en{Cisco IOU} ήταν μια χρονοβόρα διαδικασία, καθώς οι εκδόσεις αυτές δεν είναι δημόσια διαθέσιμες δωρεάν. 
Ακόμα και αν κάποιος πληρώσει για αυτές, η εύρεση της κατάλληλης έκδοσης \en{K-9} που είναι συμβατή με το \en{GNS3} που χρησιμοποιούσα, αποδείχθηκε μια πρόκληση που μου 
κόστισε πολύτιμο χρόνο.

Η εκμάθηση προγραμματισμού με \en{Python} είναι συχνά δύσκολη, ιδιαίτερα για όσους είναι νέοι στην πληροφορική. Παρόλο που η \en{Python} είναι 
δημοφιλής σε τομείς όπως η ανάπτυξη ιστοσελίδων, η επιστήμη δεδομένων και η τεχνητή νοημοσύνη, η κατανόηση της σύνταξης, των δομών δεδομένων και των 
ελεγκτικών δομών απαιτεί χρόνο. Το να μάθει κανείς προγραμματισμό σημαίνει επίσης ότι χρειάζεται λογική και αναλυτική σκέψη, που εμπλέκει δεξιότητες επίλυσης προβλημάτων 
και κριτική σκέψη.

Παρά τα εμπόδια, κατάφερα να ολοκληρώσω την εργασία μου, και διαπίστωσα ότι η υπέρβαση των προκλήσεων με γέμισε ικανοποίηση και ενθουσιασμό. 
Η δημιουργία της δικτυακής τοπολογίας με οδήγησε στο να μελετήσω περισσότερο για την ασφάλεια των θυρών και τα δυναμικά πρωτόκολλα δρομολόγησης, 
διευρύνοντας έτσι τις γνώσεις μου στα συστήματα δικτύων. Η αντιμετώπιση προβλημάτων στις διαμορφώσεις των συσκευών με βοήθησε να ανακεφαλαιώσω όσα έχω μάθει.

Η μελέτη αυτοματοποίησης δικτύου με \en{Python} μπορεί να είναι καθοριστική για άτομα στον χώρο της δικτύωσης. 
Καθώς πολλές εταιρείες υιοθετούν λύσεις αυτοματοποίησης και δικτύωσης οριζόμενης από λογισμικό (\en{SDN}), η ζήτηση για μηχανικούς δικτύων με δεξιότητες αυτοματοποίησης αυξάνεται. 
Η \en{Python}, μια γλώσσα που προσφέρει εργαλεία και βιβλιοθήκες για αυτοματοποίηση, είναι πλέον η πιο προτιμώμενη γλώσσα για τον σκοπό αυτό.

Με τη μελέτη της αυτοματοποίησης δικτύου στην \en{Python}, μπορώ να βελτιώσω τις ικανότητές μου με πολλούς τρόπους. 
Πρώτον, μου επιτρέπει να αυτοματοποιώ επαναλαμβανόμενες εργασίες δικτύου, όπως η διαμόρφωση συσκευών, η διαχείριση υποδομών δικτύου και η παρακολούθηση της απόδοσης του 
δικτύου, εξοικονομώντας χρόνο και μειώνοντας τον κίνδυνο ανθρώπινων λαθών. Δεύτερον, με βοηθά να κατανοήσω καλύτερα τα πρωτόκολλα και τις τεχνολογίες δικτύων. 
Μέσω ανάπτυξης σεναρίων και εργαλείων αυτοματοποίησης, μπορώ να αποκτήσω βαθύτερη γνώση του τρόπου λειτουργίας των δικτύων και των διαφορών που υπάρχουν στα πρωτόκολλα.

Η \en{Python} παρέχει εξαιρετική ευελιξία και προσαρμοστικότητα, επιτρέποντάς μου να αναπτύξω λύσεις αυτοματοποίησης προσαρμοσμένες στις ιδιαίτερες 
ανάγκες του οργανισμού μου. Αυτή η προσέγγιση απαιτεί βαθιά κατανόηση της αρχιτεκτονικής του δικτύου, των πρωτοκόλλων, καθώς και των θεμελιωδών αρχών του προγραμματισμού, 
ώστε να διασφαλιστεί η ακρίβεια και η αποδοτικότητα των αυτοματοποιημένων διαδικασιών.

Επιπλέον, η εργασία διατίθεται ως εφαρμογή ανοιχτού κώδικα στο \en{GitHub}, γεγονός που ενθαρρύνει τη συνέχιση της ανάπτυξης και βελτίωσής της από την κοινότητα. 
Αυτό το χαρακτηριστικό ενισχύει τη συνεργατικότητα και την ανταλλαγή ιδεών, καθιστώντας δυνατή την εξέλιξή της μέσα από την ενεργή συμμετοχή άλλων ενδιαφερόμενων προγραμματιστών.

Όπως αναφέρεται και στην εισαγωγή της διπλωματικής, η εφαρμογή έχει δυνατότητες περαιτέρω ανάπτυξης, χωρίς περιορισμούς. 
Στόχος ήταν να δημιουργηθεί ένα εργαλείο που, βασισμένο σε βασικές αρχές της επιστήμης των υπολογιστών, συνδυάζει διαφορετικές διαστάσεις της επιστήμης, 
προωθώντας την καινοτομία στη δικτύωση και την αυτοματοποίηση. Αυτή η προσέγγιση προσδίδει στην εφαρμογή έναν δυναμικό χαρακτήρα, επιτρέποντας της να αναπτυχθεί στο μέλλον 
με νέες λειτουργίες και δυνατότητες. 

