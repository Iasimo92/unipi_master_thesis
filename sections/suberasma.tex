
\chapter{Συμπέρασματα και Μελλοντική Εργασία}

\section{Συμπεράσματα}

Η διπλωματική αυτή εργασία επικεντρώθηκε στην ανάπτυξη ενός περιβάλλοντος διαχείρισης δικτυακών συσκευών µε χρήση της γλώσσας προγραµµατισµού \en{Python} και του \en{Django framework}. Κατά τη διάρκεια αυτής της διαδικασίας, αναπτύχθηκαν και εφαρμόστηκαν διάφορες τεχνολογίες για την αυτοματοποίηση δικτυακών εργασιών, όπως η διαμόρφωση στοιχείων των δικτυακών συσκευών καθώς και η παρακολούθηση πληροφοριών αυτών. Το εργαλείο αναπτύχθηκε και δοκιμάστηκε κάνοντας χρήση ενός περιβάλλοντος δοκιμών δημιουργημένου με το \en{GNS3} και με χρήση \en{Cisco IOUs}, που παρείχε την δυνατότητα ελέγχων της εφαρμογής κατά την διάρκεια της ανάπτυξής της.

\section{Προκλήσεις και Μαθήματα}


Οι προκλήσεις που αντιμετωπίστηκαν κατά τη διάρκεια της 
εργασίας περιελάμβαναν τη διαχείριση περιορισμών, όπως η 
εύρεση της κατάλληλης έκδοσης \en{Cisco IOU}, μια διαδικασία 
χρονοβόρα λόγω της μη δημόσιας διαθεσιμότητάς τους. 
Αυτό απαίτησε χρόνο για έρευνα και επίλυση τεχνικών προβλημάτων, 
οδηγώντας σε καθυστερήσεις, αλλά και σε πολύτιμα μαθήματα για τη 
διαχείριση κρίσιμων πόρων και την επιμονή στις δυσκολίες. Η εκμάθηση 
της \en{Python} υπήρξε εξίσου απαιτητική, καθώς απαιτούσε 
εμβάθυνση στη σύνταξη, στις δομές δεδομένων και στον προγραμματισμό. 
Οι δυσκολίες αυτές έγιναν ευκαιρίες κατανόησης της δύναμης της 
γλώσσας, ειδικά στον τομέα της αυτοματοποίησης. Η εμπειρία αυτή 
ανέδειξε τη σημασία της κριτικής σκέψης και της δημιουργίας 
ρεαλιστικών λύσεων.

Τέλος, η ανάπτυξη της εφαρμογής έδειξε πώς οι 
τεχνικές προκλήσεις μπορούν να μετατραπούν σε μαθήματα ζωής, 
όπως η διαχείριση χρόνου, η συνεργασία με εργαλεία ανοιχτού 
κώδικα και η ανάγκη για συνεχή ενημέρωση σε νέες τεχνολογίες. 
Η εργασία ανέδειξε τη σημασία της προσαρμοστικότητας και της 
δια βίου μάθησης στον τομέα των δικτύων

\section{Μελλοντική Εργασία και Επέκταση Λειτουργικότητας}

Η μελλοντική εργασία θα επικεντρωθεί στην περαιτέρω 
βελτίωση της λειτουργικότητας της εφαρμογής, ενσωματώνοντας 
επιπλέον πρωτόκολλα δικτύωσης και αυτοματοποίησης. 
Μια πιθανή κατεύθυνση είναι η ανάπτυξη μηχανισμών για την 
υποστήριξη \en{SDN} (δικτύωση οριζόμενη από λογισμικό), 
προκειμένου να ανταποκριθεί στις απαιτήσεις σύγχρονων δικτύων.

Επίσης, θα μπορούσαν να προστεθούν χαρακτηριστικά όπως 
η παρακολούθηση της απόδοσης του δικτύου σε πραγματικό 
χρόνο και η αυτοματοποίηση διαδικασιών αποκατάστασης προβλημάτων. 
Οι επεκτάσεις θα ενισχύσουν την ευελιξία και τη χρηστικότητα του 
εργαλείου.

Το περιβάλλον της εφαρμογής θα μπορούσε να είναι διαφορετικό.
Μελλοντική εργασία μπορεί να αναπτυχθεί με τη βοήθεια καινούργιων περιβάλλοντων προσομοιώσης όπως
το \en{ContainerLab}. Η περαιτέρω ανάπτυξη λειτουργιών της εφαρμογής, η δημιουργία καλυτερου \en{User Experience}
και \en{User Interface} καθώς και η δημιουργία \en{Testing Platform} για την αυτοματοποίηση
του \en{testing} της εφαρμογής θα μπορούσαν να αποτελέσουν ξεχωριστό επίσης θέμα για διπλωματική εργασία.

Τέλος, η μελλοντική έρευνα μπορεί να εξετάσει τη δυνατότητα διασύνδεσης με 
πλατφόρμες μηχανικής μάθησης, για την 
πρόβλεψη και αποτροπή πιθανών αποτυχιών, 
καθιστώντας την εφαρμογή ένα σύγχρονο εργαλείο 
αυτοματοποιημένης διαχείρισης δικτύου