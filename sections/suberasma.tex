
\chapter{Συμπέρασματα και Μελλοντική Εργασία}

\section{Συμπεράσματα}

Η διπλωματική αυτή εργασία επικεντρώθηκε στην ανάπτυξη ενός εργαλείου αυτοματοποίησης 
δικτύου με χρήση της γλώσσας προγραμματισμού \en{Python}. 
Κατά τη διάρκεια αυτής της διαδικασίας, καταφέραμε να αναπτύξουμε και να 
εφαρμόσουμε διάφορες τεχνολογίες για την αυτοματοποίηση δικτυακών εργασιών, 
όπως η διαμόρφωση συσκευών και η παρακολούθηση της απόδοσης του δικτύου. 
Το εργαλείο αναπτύχθηκε χρησιμοποιώντας σαν \en{TestBed} το \en{GNS3} και το \en{Cisco IOU}, 
και παρέχει σημαντικά οφέλη στον τομέα της δικτύωσης.


Η εφαρμογή αυτή έχει σημαντική χρησιμότητα στον τομέα της δικτύωσης, 
κυρίως για οργανισμούς που αναζητούν λύσεις αυτοματοποίησης και 
διαχείρισης δικτύων με σκοπό τη βελτίωση της απόδοσης και την 
εξοικονόμηση χρόνου. Μπορεί να εφαρμοστεί σε διάφορα περιβάλλοντα και 
να επεκταθεί για να καλύψει νέες ανάγκες, όπως η ενσωμάτωση νέων πρωτοκόλλων 
και εργαλείων παρακολούθησης.

Όπως είπαμε στην αρχή οι τεχνολογίες \en{Cloud Native} δεν εισάχθην προκειμένου
να γίνουν μετρήσεις ταχύτητας μεταξύ της εφαρμογής σε περιβάλλον τοπικό και σε περιβάλλον κυβερνήτη.
Η εισαγωγή αυτής της τεχνολογίας έγινε προκειμένου και μόνο να μπορούμε να διαχειριστούμε την εφαρμογή
ως προς τα πλαίσια της κλιμάκωσης και της φορητότητάς της. Συνεπώς δεν ήταν στόχος μας να παρουσιαστούν πολιτικές
εκτενείς διαφορές και πειραματισμοί κάτι το οποίο από μόνο του θα μπορούσε να αποτελέσει ένα άλλο θέμα για
διπλωματική διατριβή μεταπτυχιακού επιπέδου.

\section{Προκλήσεις και Μαθήματα}


Οι προκλήσεις που αντιμετωπίστηκαν κατά τη διάρκεια της 
εργασίας περιελάμβαναν τη διαχείριση περιορισμών, όπως η 
εύρεση της κατάλληλης έκδοσης \en{Cisco IOU}, μια διαδικασία 
χρονοβόρα λόγω της μη δημόσιας διαθεσιμότητάς τους. 
Αυτό απαίτησε χρόνο για έρευνα και επίλυση τεχνικών προβλημάτων, 
οδηγώντας σε καθυστερήσεις, αλλά και σε πολύτιμα μαθήματα για τη 
διαχείριση κρίσιμων πόρων και την επιμονή στις δυσκολίες. Η εκμάθηση 
της \en{Python} υπήρξε εξίσου απαιτητική, καθώς απαιτούσε 
εμβάθυνση στη σύνταξη, στις δομές δεδομένων και στον προγραμματισμό. 
Οι δυσκολίες αυτές έγιναν ευκαιρίες κατανόησης της δύναμης της 
γλώσσας, ειδικά στον τομέα της αυτοματοποίησης. Η εμπειρία αυτή 
ανέδειξε τη σημασία της κριτικής σκέψης και της δημιουργίας 
ρεαλιστικών λύσεων.

Τέλος, η ανάπτυξη της εφαρμογής έδειξε πώς οι 
τεχνικές προκλήσεις μπορούν να μετατραπούν σε μαθήματα ζωής, 
όπως η διαχείριση χρόνου, η συνεργασία με εργαλεία ανοιχτού 
κώδικα και η ανάγκη για συνεχή ενημέρωση σε νέες τεχνολογίες. 
Η εργασία ανέδειξε τη σημασία της προσαρμοστικότητας και της 
δια βίου μάθησης στον τομέα των δικτύων

\section{Μελλοντική Εργασία και Επέκταση Λειτουργικότητας}

Η μελλοντική εργασία θα επικεντρωθεί στην περαιτέρω 
βελτίωση της λειτουργικότητας της εφαρμογής, ενσωματώνοντας 
επιπλέον πρωτόκολλα δικτύωσης και αυτοματοποίησης. 
Μια πιθανή κατεύθυνση είναι η ανάπτυξη μηχανισμών για την 
υποστήριξη \en{SDN} (δικτύωση οριζόμενη από λογισμικό), 
προκειμένου να ανταποκριθεί στις απαιτήσεις σύγχρονων δικτύων.

Επίσης, θα μπορούσαν να προστεθούν χαρακτηριστικά όπως 
η παρακολούθηση της απόδοσης του δικτύου σε πραγματικό 
χρόνο και η αυτοματοποίηση διαδικασιών αποκατάστασης προβλημάτων. 
Οι επεκτάσεις θα ενισχύσουν την ευελιξία και τη χρηστικότητα του 
εργαλείου.

Το περιβάλλον πάνω στο οποίο τεσταρίστηκε η εφαρμογή θα μπορούσε να είναι διαφορετικό.
Μελλοντική εργασία μπορεί να αναπτυχθεί με τη βοήθεια καινούργιων περιβάλλοντων προσομοιώσης όπως
το \en{ContainerLab}. Η περαιτέρω ανάπτυξη λειτουργιών της εφαρμογής, η δημιουργία καλυτερου \en{User Experience}
και \en{User Interface} καθώς και η δημιουργία \en{Testing Platform} για την αυτοματοποίηση
του \en{testing} της εφαρμογής θα μπορούσαν να αποτελέσουν ξεχωριστό επίσης θέμα για διπλωματική εργασία.

Τέλος, η μελλοντική έρευνα μπορεί να εξετάσει τη δυνατότητα διασύνδεσης με 
πλατφόρμες μηχανικής μάθησης, για την 
πρόβλεψη και αποτροπή πιθανών αποτυχιών, 
καθιστώντας την εφαρμογή ένα σύγχρονο εργαλείο 
αυτοματοποιημένης διαχείρισης δικτύου