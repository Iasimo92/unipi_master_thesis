\chapter{\en{Virtual Environment Set up}}

\section{\en{GNS3 Installation}}

Το \en{GNS3} είναι ένα λογισμικό που χρησιμοποιείται για την εξομοίωση, τη διαμόρφωση και τη δοκιμή ενός περιβάλλοντος δικτύου. Είναι
είναι ένα ελεύθερο λογισμικό ανοικτού κώδικα και μπορείτε να το κατεβάσετε από τον επίσημο δικτυακό τόπο 
\en{https://www.gns3.com/} .Το \en{GNS3} αποτελείται από δύο στοιχεία. Το ολοκληρωμένο λογισμικό (\en{GUI}) το οποίο είναι ένα γραφικό 
διεπαφή χρήστη και την εικονική μηχανή (\en{VM}), η οποία είναι ένας διακομιστής που εκτελείται σε εικονικό περιβάλλον και παρέχει καλύτερο μέγεθος τοπολογίας και υποστήριξη συσκευών.
Η εγκατάσταση είναι απλή και θα πρέπει να χρησιμοποιούνται οι προεπιλεγμένες επιλογές.

Για να γίνει σωστά η εγκατάσταση θα πρέπει το \en{software version} του \en{GNS3} να είναι το ίδιο με το
\en{software version} του \en{GNS3 VM}. Όταν λοιπόν γίνει η εγκατάσταση και ανοίγουμε το \en{GNS3 GUI}
αυτή η ενέργεια θα κάνει \en{trigger} το \en{booting} του \en{GNS3 VM}
Μόλις γίνει η εγκατάσταση μπορεί να ανοίξει η εφαρμογή και να κάνουμε \en{import cisco IOS images}. Στην παρακάτω
εικόνα μπορούμε να δούμε τι γίνεται όταν ανοίγουμε το \en{GNS3}. 

\begin{figure}[htb]
	\centering
	\includegraphics[width=0.7\textwidth]{graphics/gns3_homepage.png}
	\caption{\en{GNS3 homepage} }
\end{figure}

Προκειμένου να μπορέσει να επικοινωνήσει το \en{PC} μας στο τοπικό δίκτυο με το \en{GNS3 VM} στο τοπικό δίκτυο
θα πρέπει να γίνουν κάποιες ρυθμίσεις τόσο στο \en{GNS3 VM} όσο και στις συσκευές της \en{Cisco}

Στις συσκευές της \en{Cisco} θα πρέπει να γίνει η παρακάτω παραμετροποίηση όπως εμφανίζεται στις εικόνες 4.1,4.2,4.3.

\begin{figure}[htb]
	\centering
	\includegraphics[width=0.7\textwidth]{graphics/cisco_ssh_config.png}
	\caption{\en{Cisco ssh config} }
\end{figure}

\begin{figure}[htb]
	\centering
	\includegraphics[width=0.7\textwidth]{graphics/dhcp_cisco_config.png}
	\caption{\en{Cisco dhcp config} }
\end{figure}


Μέχρι αυτή τη στιγμή έχουμε παραμετροποιήσει τις συσκευές με τέτοιο τρόπο ώστε να δέχονται
απομακρυσμένη σύνδεση. Τώρα θα εξηγήσουμε πως μπορούμε να φτιάξουμε την εποικοινωνία μεταξύ εικονικών
μηχανών της \en{Cisco} και του τοπικού μας υπολογιστή. Η λογική είναι ότι η συσκευή \en{Cloud}
θα μας επιτρέψει να φτιάξουμε τη σύνδεση αυτή. Η εικόνα 4.4 μας παρουσιάζει σε ανώτερο επίπεδο τη λογική αυτή σύνδεση.

\begin{figure}[htb]
	\centering
	\includegraphics[width=0.7\textwidth]{graphics/diagram.drawio.png}
	\caption{\en{Local PC-GNS3VM-CISCO IOS Connection Architecture} }
\end{figure}









\section{\en{Connection Establishment} }


Όταν λοιπόν γίνει αυτή η παραμετροποίηση και τοπολογία θα πρέπει όλα αυτά τα διαφορετικά components να ανήκουν στο ίδιο τοπικό δίκτυο.
Η εικονική διεπαφή από την οποία θα περνάει όλη η κίνηση είτε μιλάμε για \en{REST} ειτε για \en{SSH} θα είναι η \en{eth0} στο \en{GNS3 VM}. Παρακάτω ένα 
\en{trace} στην εικόνα 4.5 που συλλέχτηκε αποδεικνύει ότι η σύνδεση πραγματοποιείται χωρίς προβλήματα.

\begin{figure}[htb]
	\centering
	\includegraphics[width=0.7\textwidth]{graphics/ssh_connection.png}
	\caption{\en{SSH traffic} }
\end{figure}


Προκειμένου να γίνει η συλλογή του συγκεκριμένου \en{trace} χρησιμοποιήθηκε η παρακάτω εντολή:
\en{tcpdump -i eth0 -v -w /home/gns3/test.pcap}.Η συλλογή του \en{trace} έγινε με το πρωτόκολλο \en{SFTP}.


\begin{figure}[htb]
	\centering
	\includegraphics[width=0.7\textwidth]{graphics/jason1.png}
	\caption{\en{SSH traffic} }
\end{figure}