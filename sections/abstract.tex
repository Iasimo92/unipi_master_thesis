\chapter*{Περίληψη}

Περίληψη διπλωματικής εργασίας.\\

\noindent Η παρούσα διπλωματική εργασία επικεντρώνεται στη μελέτη και ανάπτυξη λογισμικού για την παραμετροποίηση δικτυακών συσκευών, 
αξιοποιώντας το πλαίσιο \en{Django} της γλώσσας προγραμματισμού \en{Python}. Η ιδέα για την υλοποίηση αυτής της εφαρμογής προέκυψε από το ενδιαφέρον 
για τους τομείς του \en{software development}, \en{networking},\en{SDN} και \en{Cloud Native.} εμπνευσμένη από παρόμοιες εφαρμογές, όπως αυτή ενός μηχανικού της \en{Cisco}, 
καθώς και από σχετικές δημοσιεύσεις στον τομέα.

Σκοπός μας ήταν να μεταφέρουμε και να εξελίξουμε αυτές τις ιδέες, ενσωματώνοντας νέες έννοιες, όπως τα \en{microservices} και το \en{Kubernetes}, 
για να επιτύχουμε έναν συνδυασμό παραμετροποίησης δικτύων και αυτοματοποίησης που να ανταποκρίνεται στις απαιτήσεις σύγχρονων υποδομών.

Μέσω της παρούσας εφαρμογής, επιχειρούμε να συνδυάσουμε γνώσεις από διάφορους τομείς της Πληροφορικής, συνδυάζοντας την αυτοματοποίηση με 
καινοτόμες τεχνολογίες. Στο τέλος της εργασίας, θα παρουσιαστεί εκτενής βιβλιογραφία, συμπεριλαμβάνοντας τις πηγές που μας ενέπνευσαν κατά την 
ανάπτυξη αυτής της εφαρμογής. Η αυτοματοποίηση αποτελεί ακρογωνιαίο λίθο της σύγχρονης Πληροφορικής, και η εργασία αυτή φιλοδοξεί να αναδείξει τον 
τρόπο με τον οποίο οι εταιρείες υιοθετούν αυτές τις τεχνολογίες για να καινοτομήσουν και να αναπτύξουν τα προϊόντα τους, ακολουθώντας τις εξελίξεις στον τομέα.

Η διπλωματική αυτή εργασία εκπονήθηκε σε μία από τις πιο γόνιμες περιόδους της ακαδημαϊκής και επαγγελματικής μου πορείας. 
Κατά τη χρονιά που ξεκίνησα το μεταπτυχιακό μου το 2021, είχα ήδη συγκεντρώσει δύο χρόνια εμπειρίας ως μηχανικός δικτύωσης και λογισμικού στον τομέα των 
τηλεπικοινωνιών. Η εμπειρία αυτή αποτέλεσε τη βάση για την απόφασή μου να διευρύνω τις γνώσεις μου και να εμβαθύνω σε θέματα που αφορούν τις σύγχρονες 
τεχνολογικές εξελίξεις στον χώρο των τηλεπικοινωνιών. Το επαγγελματικό μου υπόβαθρο συνέβαλε στην κατανόηση των θεωρητικών εννοιών, αλλά και στη διαμόρφωση 
του ερευνητικού προσανατολισμού της διπλωματικής μου.

Η αγάπη μου για τον κλάδο της τεχνολογίας και της μηχανικής δικτύωσης αποτέλεσε το κύριο κίνητρο για την εκπόνηση 
αυτού του μεταπτυχιακού προγράμματος, και ιδιαίτερα αυτής της διπλωματικής εργασίας. Επιδίωξή μου ήταν να κατανοήσω σε βάθος τις προκλήσεις και τις 
καινοτομίες που χαρακτηρίζουν τον τομέα των τηλεπικοινωνιών, με στόχο να εξελιχθώ επαγγελματικά και να συνεισφέρω ουσιαστικά στις τεχνολογικές εξελίξεις. 
Η συγκεκριμένη διπλωματική εργασία μου έδωσε τη δυνατότητα να εξερευνήσω σημαντικές θεματικές, όπως την ασφάλεια δικτύων, τη διαχείριση δεδομένων και την 
ανάπτυξη λογισμικού, που αποτελούν κρίσιμους παράγοντες για την παροχή αποτελεσματικών και αξιόπιστων υπηρεσιών τηλεπικοινωνίας.

Μέσα από τη συγγραφή αυτής της εργασίας, δεν απέκτησα μόνο θεωρητικές γνώσεις, αλλά ανέπτυξα και πρακτικές δεξιότητες που εμπλούτισαν την 
επαγγελματική μου ταυτότητα. Η προσέγγιση της διπλωματικής συνδυάζει την εμπειρία μου ως μηχανικός με την έρευνα, προσφέροντάς μου την ευκαιρία να 
αντιμετωπίσω σύνθετα προβλήματα με κριτική σκέψη και δημιουργικότητα. Επομένως, η εργασία αυτή δεν είναι απλώς ένα ακαδημαϊκό επίτευγμα αλλά και μια 
επένδυση που με εφοδιάζει με τις γνώσεις και τις ικανότητες που απαιτούνται για μια επιτυχημένη καριέρα στον χώρο της τεχνολογίας και των τηλεπικοινωνιών.