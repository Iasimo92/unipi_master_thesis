\chapter{Πρόλογος}

\section{Σκοπός και στόχοι της διπλωματικής εργασίας}

Η πτυχιακή αυτή εργασία έχει ως σκοπό την ανάπτυξη μιας σύγχρονης εφαρμογής για την παραμετροποίηση δικτυακών συσκευών, αξιοποιώντας τις δυνατότητες που προσφέρουν τα σύγχρονα τεχνολογικά εργαλεία και πρότυπα. Η επιλογή του συγκεκριμένου θέματος βασίζεται στη διαρκώς αυξανόμενη ανάγκη για αυτοματοποίηση και ευελιξία στη διαχείριση δικτύων, ιδιαίτερα σε περιβάλλοντα που χαρακτηρίζονται από μεγάλη κλίμακα και πολυπλοκότητα.

Οι βασικοί στόχοι της εργασίας περιλαμβάνουν:

\begin{itemize}
    \item Τη δημιουργία μιας φιλικής προς τον χρήστη εφαρμογής που θα απλοποιεί τη διαδικασία παραμετροποίησης δικτυακών συσκευών.
    \item Τη χρήση σύγχρονων τεχνολογιών, όπως τα \en{microservices} και τα \en{containers}, για την εξασφάλιση κλιμακωσιμότητας και φορητότητας.
    \item Την ενσωμάτωση εργαλείων αυτοματοποίησης και τη βελτιστοποίηση των διαδικασιών διαχείρισης.
    \item Την αξιολόγηση της εφαρμογής σε προσομοιωμένα περιβάλλοντα για την επιβεβαίωση της λειτουργικότητας και της απόδοσής της.
\end{itemize}


Η εργασία στοχεύει να προσφέρει μια ολοκληρωμένη λύση που θα συνδυάζει καινοτομία, πρακτικότητα και δυνατότητες για μελλοντική επέκταση.

\section{Συνοπτική Περιγραφή της Εφαρμογής και της Υλοποίησης}

Η εφαρμογή που αναπτύχθηκε βασίζεται στο \en{Django framework} της γλώσσας προγραμματισμού \en{Python}, παρέχοντας ένα ολοκληρωμένο \en{backend} σύστημα για τη διαχείριση και παραμετροποίηση δικτυακών συσκευών.

Για την υλοποίηση και τη δοκιμή της εφαρμογής:

\begin{itemize}
    \item Χρησιμοποιήθηκαν εργαλεία όπως το \en{Docker} για την ανάπτυξη και τη διαχείριση των \en{containers}, εξασφαλίζοντας φορητότητα και σταθερότητα.
    \item Δοκιμάστηκε σε περιβάλλον \en{GNS3}, όπου προσομοιώθηκαν διάφορες συνθήκες δικτύου για την επιβεβαίωση της λειτουργικότητας της εφαρμογής.
    \item Εφαρμόστηκαν τεχνολογίες όπως τα \en{RESTful APIs} και το \en{SSH} πρωτοκολλο για τη διασύνδεση με τις δικτυακές συσκευές, ενώ η αρχιτεκτονική της εφαρμογής βασίζεται σε μικροϋπηρεσίες για μεγαλύτερη ευελιξία και επεκτασιμότητα.
\end{itemize}

Η διαδικασία ανάπτυξης περιλάμβανε τη σχεδίαση ενός περιβάλλοντος φιλικού προς τον χρήστη για την εισαγωγή και επεξεργασία ρυθμίσεων, την ενσωμάτωση εργαλείων για αυτοματοποιημένη εκτέλεση εντολών και την αξιοποίηση τεχνολογιών \en{containerization} για την εύκολη ανάπτυξη της εφαρμογής σε διαφορετικά περιβάλλοντα.
Η εφαρμογή φιλοδοξεί να αποτελέσει ένα χρήσιμο εργαλείο για προγραμματιστές που ενδιαφέρονται να αναπτύξουν παρόμοιες εφαρμογές καθώς και σαν γενικότερη συμβολή στο χώρο της αυτοματοποίησης και του προγραμματισμού.