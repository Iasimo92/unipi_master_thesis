\chapter*{Περίληψη}

Περίληψη διπλωματικής εργασίας.\\

\noindent Η παρούσα διπλωματική εργασία έχει ως στόχο τη μελέτη και την ανάπτυξη λογισμικού για την παραμετροποίηση δικτυακών συσκευών, χρησιμοποιώντας ως βασικό πλαίσιο το \en{Django} της \en{Python}. 
Η ιδέα για την υλοποίηση αυτής της εφαρμογής προέκυψε από το υπόβαθρό μας στον τομέα του \en{software development} και του \en{networking},
με έμπνευση από μία εφαρμογή ενός μηχανικού της \en{Cisco}, 
καθώς και από διάφορες σχετικές δημοσιεύσεις. Στόχος μας ήταν να πάρουμε αυτές τις ιδέες και να τις εξελίξουμε, εισάγοντας νέες έννοιες όπως τα 
\en{microservices} και το \en{Kubernetes}.

Μέσω της εφαρμογής αυτής, επιδιώκουμε να ενσωματώσουμε γνώσεις από πολλούς τομείς της Πληροφορικής, συνδυάζοντας την αυτοματοποίηση με σύγχρονες τεχνολογίες. 
Θα παρουσιαστεί εκτενής βιβλιογραφία στο τέλος της εργασίας, η οποία θα περιλαμβάνει αναφορές στις πηγές που μας ενέπνευσαν. Η αυτοματοποίηση αποτελεί θεμέλιο λίθο στη σύγχρονη Πληροφορική, και η εργασία αυτή επιχειρεί να παρουσιάσει πώς κινούνται οι εταιρείες στον χώρο, προσπαθώντας να καινοτομήσουν και να αναπτύξουν τα προϊόντα τους, με βάση τις τελευταίες εξελίξεις της τεχνολογίας.