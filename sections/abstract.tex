\chapter*{Περίληψη}

Περίληψη διπλωματικής εργασίας.\\

\noindent Η παρούσα διπλωματική εργασία επικεντρώνεται στη μελέτη και ανάπτυξη λογισμικού για την παραμετροποίηση δικτυακών συσκευών, 
αξιοποιώντας το πλαίσιο \en{Django} της γλώσσας προγραμματισμού \en{Python}. Η ιδέα για την υλοποίηση αυτής της εφαρμογής προέκυψε από το ενδιαφέρον 
για τους τομείς του \en{software development}, \en{networking},\en{SDN} και \en{Cloud Native.} εμπνευσμένη από παρόμοιες εφαρμογές, όπως αυτή ενός μηχανικού της \en{Cisco}, 
καθώς και από σχετικές δημοσιεύσεις στον τομέα.

Σκοπός μας ήταν να μεταφέρουμε και να εξελίξουμε αυτές τις ιδέες, ενσωματώνοντας νέες έννοιες, όπως τα \en{microservices} και το \en{Kubernetes}, 
για να επιτύχουμε έναν συνδυασμό παραμετροποίησης δικτύων και αυτοματοποίησης που να ανταποκρίνεται στις απαιτήσεις σύγχρονων υποδομών.

Μέσω της παρούσας εφαρμογής, επιχειρούμε να συνδυάσουμε γνώσεις από διάφορους τομείς της Πληροφορικής, συνδυάζοντας την αυτοματοποίηση με 
καινοτόμες τεχνολογίες. Στο τέλος της εργασίας, θα παρουσιαστεί εκτενής βιβλιογραφία, συμπεριλαμβάνοντας τις πηγές που μας ενέπνευσαν κατά την 
ανάπτυξη αυτής της εφαρμογής. Η αυτοματοποίηση αποτελεί ακρογωνιαίο λίθο της σύγχρονης Πληροφορικής, και η εργασία αυτή φιλοδοξεί να αναδείξει τον 
τρόπο με τον οποίο οι εταιρείες υιοθετούν αυτές τις τεχνολογίες για να καινοτομήσουν και να αναπτύξουν τα προϊόντα τους, ακολουθώντας τις εξελίξεις στον τομέα.