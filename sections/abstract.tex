\chapter{Περίληψη}
\section{Περίληψη}

Η πτυχιακή εργασία αφορά την ανάπτυξη μιας εφαρμογής για την παραμετροποίηση δικτυακών συσκευών, χρησιμοποιώντας το \en{Django} και βιβλιοθήκες \en{Python} για αυτοματοποίηση. Η εφαρμογή απλοποιεί τη διαδικασία παραμετροποίησης μέσω ενός φιλικού περιβάλλοντος και αυτοματοποιημένων διαδικασιών. Εφαρμόζονται αρχές \en{DevOps} όπως \en{CI/CD}, με τη χρήση τεχνολογιών όπως το \en{Netmiko} και το \en{GNS3} για δοκιμές σε ρεαλιστικές συνθήκες. Παρουσιάζονται τα βήματα ανάπτυξης, οι τεχνολογικές προκλήσεις και οι λύσεις που υιοθετήθηκαν, προσφέροντας μια πρακτική και επεκτάσιμη εφαρμογή.

\vspace{0.3cm}  % Ρύθμιση για να μειώσουμε το διάστημα μετά την περίληψη

\noindent
\begin{minipage}[t]{\textwidth}  % Χρησιμοποιούμε minipage για να διασφαλίσουμε ότι δεν θα περάσει σε νέα σελίδα
    \vspace{0cm}  % Καταργούμε το αυτόματο κενό που μπορεί να προσθέσει ο LaTeX
    
    \begin{abstract}
    \selectlanguage{english}
    \section*{Abstract}

    This thesis focuses on the development of an application for the configuration of network devices, using Django and Python libraries for automation. The application simplifies the configuration process through a user-friendly interface and automated procedures. DevOps principles like CI/CD are applied, using technologies such as Netmiko and GNS3 for testing in real-world conditions. The thesis presents the development steps, technological challenges, and solutions adopted, offering a practical and scalable application.
    \end{abstract}
    \selectlanguage{greek}
    
\end{minipage}

\clearpage

\section{Συνοπτική Περιγραφή της Εργασίας}

Η παρούσα πτυχιακή εργασία αποσκοπεί στην ανάπτυξη μιας σύγχρονης 
εφαρμογής για την παραμετροποίηση δικτυακών συσκευών, αξιοποιώντας 
τις δυνατότητες που προσφέρουν τα σύγχρονα τεχνολογικά εργαλεία, 
πρότυπα και αρχές \en{DevOps}. 
Η επιλογή του θέματος βασίζεται στη διαρκώς αυξανόμενη ανάγκη 
για αυτοματοποιημένες, επεκτάσιμες και αποδοτικές λύσεις διαχείρισης, 
ειδικά σε περιβάλλοντα με υψηλή κλίμακα και πολυπλοκότητα.

Η εφαρμογή αναπτύχθηκε με τη χρήση του \en{Django framework} της 
γλώσσας προγραμματισμού \en{Python}, προσφέροντας ένα ολοκληρωμένο \en{backend} 
σύστημα για τη διαχείριση και παραμετροποίηση δικτυακών συσκευών. 

Η ανάπτυξη υιοθέτησε τις αρχές του \en{DevOps}, 
διασφαλίζοντας τη συνεχή ενσωμάτωση και παράδοση, 
καθώς και την αποτελεσματική παρακολούθηση και 
υποστήριξη της εφαρμογής. Παρόλο που η εργασία δεν περιλαμβάνει τη 
ρύθμιση εργαλείων όπως το \en{Jenkins} ή το \en{GitHub Actions} 
για τη συνεχή παράδοση – καθώς αυτό γίνεται με μη αυτοματοποιημένο τρόπο στη δική μας περίπτωση – 
οι ενσωματωμένες πρακτικές του \en{DevOps} για τη συνεχή 
ενσωμάτωση επιτρέπουν την αυτοματοποίηση της ροής εργασιών. 
Επιπλέον, διευκολύνουν την άμεση ενημέρωση της ομάδας για αλλαγές 
στον κώδικα, ενισχύοντας τη συνεργασία και τη διαφάνεια μέσω της 
χρήσης του \en{GitHub}

Κατά την υλοποίηση και δοκιμή της εφαρμογής χρησιμοποιήθηκαν εργαλεία όπως το \en{Docker}, 
που εξασφαλίζει φορητότητα και σταθερότητα μέσω της ανάπτυξης και διαχείρισης \en{containers}. 
Παράλληλα η εισαγωγή του κυβερνήτη ως τεχνολογία διαχείρισης κοντέινερς αξιοποιήθηκε 
για να επιδείξει πώς μια εφαρμογή μπορεί να λειτουργεί αποτελεσματικά σε ένα τέτοιο περιβάλλον, 
αναδεικνύοντας την επεκτασιμότητα και την ευελιξία της σε πραγματικές συνθήκες

Η εφαρμογή δοκιμάστηκε σε προσομοιωμένο περιβάλλον \en{GNS3}, 
όπου αναπαραστάθηκαν διαφορετικές συνθήκες δικτύου για την 
αξιολόγηση της λειτουργικότητας και της απόδοσής της.

Επιπλέον, αξιοποιήθηκαν τεχνολογίες όπως τα \en{RESTful APIs} 
και το πρωτόκολλο \en{SSH} για τη διασύνδεση με δικτυακές συσκευές, 
ενώ η αρχιτεκτονική της εφαρμογής βασίστηκε σε μικροϋπηρεσίες 
για να διασφαλιστεί η ευελιξία και η επεκτασιμότητα. 

Η ανάπτυξη περιλάμβανε τη δημιουργία ενός φιλικού προς τον χρήστη 
περιβάλλοντος για την εισαγωγή και επεξεργασία ρυθμίσεων, καθώς και 
την ενσωμάτωση εργαλείων για την αυτοματοποιημένη εκτέλεση εντολών. 
Ιδιαίτερη έμφαση δόθηκε στις τεχνολογίες \en{containerization}, 
οι οποίες επιτρέπουν την εύκολη ανάπτυξη της εφαρμογής σε διαφορετικά περιβάλλοντα.

Συνολικά, η εργασία συνδυάζει πρακτικές αυτοματοποίησης και \en{DevOps} 
με τη χρήση σύγχρονων τεχνολογικών εργαλείων, δημιουργώντας ένα 
ολοκληρωμένο πλαίσιο διαχείρισης δικτυακών συσκευών. 
Η εφαρμογή φιλοδοξεί να αποτελέσει ένα πολύτιμο εργαλείο 
για προγραμματιστές και διαχειριστές δικτύων, συμβάλλοντας 
σημαντικά στον τομέα της αυτοματοποίησης και της παραμετροποίησης 
δικτύων.