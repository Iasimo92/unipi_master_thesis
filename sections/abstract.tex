\chapter{Πρόλογος}

\section{Περίληψη}

Η πτυχιακή αυτή εργασία αφορά την ανάπτυξη µιας σύγχρονης εφαρµογής 
για την παραµετροποίηση δικτυακών συσκευών. Η εφαρμογή αξιοποιεί το 
πλαίσιο λογισμικού \en{Django}, ενσωματώνει βιβλιοθήκες \en{Python} 
για αυτοματοποίηση και για τη δοκιμή της γίνεται χρήση ενός εικονικού
εργαστηρίου στο \en{GNS3}. 

Στόχος της εφαρμογής είναι η απλοποίηση 
της διαδικασίας παραμετροποίησης δικτυακών συσκευών, παρέχοντας στο 
χρήστη ένα φιλικό περιβάλλον εργασίας που θα επιτρέπει την 
διαχείριση και την εφαρμογή ρυθμίσεων μέσω αυτοματοποιημένων 
διαδικασιών. Κατά την ανάπτυξη της εργασίας έχουν αξιοποιηθεί αρχές 
του \en{DevOps} όπως η συνεχής ενσωμάτωση και παράδοση \en{CI/CD} 
χρησιμοποιώντας μια απλοποημένη λογική η οποία σε καμία περίπτωση 
δεν προσομοιάζει τα εργαλεία και την πολυπλοκότητα μιας μεγάλης σε κλίμακα εταιρείας. Χρησιμοποιήθηκαν τεχνολογίες όπως το \en{Netmiko} και το \en{Paramiko} για την επικοινωνία με τις συσκευές, καθώς και το \en{GNS3} ως πλατφόρμα προσομοίωσης για την αξιολόγηση της εφαρμογής σε περιβάλλον πραγματικών συνθηκών.
Μέσα από την εργασία παρουσιάζονται τα βήματα υλοποίησης της 
εφαρμογής, οι τεχνολογικές προκλήσεις που αντιμετωπίστηκαν και 
οι λύσεις που υιοθετήθηκαν. Έτσι, στην εργασία αυτή πέρα από την 
εφαρμογή που υλοποιήθηκε που συνδυάζει την πρακτικότητα και την 
επεκτασιμότητα, γίνεται και περιγραφή των διαδικασιών που 
ακολουθήκαν και αποτελούν τη βάση για την ανάπτυξη σύγχρονων 
εφαρμογών.

\section{Συνοπτική Περιγραφή της Εργασίας}

Η παρούσα πτυχιακή εργασία αποσκοπεί στην ανάπτυξη μιας σύγχρονης 
εφαρμογής για την παραμετροποίηση δικτυακών συσκευών, αξιοποιώντας 
τις δυνατότητες που προσφέρουν τα σύγχρονα τεχνολογικά εργαλεία, 
πρότυπα και αρχές \en{DevOps}. 
Η επιλογή του θέματος βασίζεται στη διαρκώς αυξανόμενη ανάγκη 
για αυτοματοποιημένες, επεκτάσιμες και αποδοτικές λύσεις διαχείρισης, 
ειδικά σε περιβάλλοντα με υψηλή κλίμακα και πολυπλοκότητα.

Η εφαρμογή αναπτύχθηκε με τη χρήση του \en{Django framework} της 
γλώσσας προγραμματισμού \en{Python}, προσφέροντας ένα ολοκληρωμένο \en{backend} 
σύστημα για τη διαχείριση και παραμετροποίηση δικτυακών συσκευών. 

Η ανάπτυξη υιοθέτησε τις αρχές του \en{DevOps}, 
διασφαλίζοντας τη συνεχή ενσωμάτωση και παράδοση, 
καθώς και την αποτελεσματική παρακολούθηση και 
υποστήριξη της εφαρμογής. Παρόλο που η εργασία δεν περιλαμβάνει τη 
ρύθμιση εργαλείων όπως το \en{Jenkins} ή το \en{GitHub Actions} 
για τη συνεχή παράδοση – καθώς αυτό γίνεται με μη αυτοματοποιημένο τρόπο στη δική μας περίπτωση – 
οι ενσωματωμένες πρακτικές του \en{DevOps} για τη συνεχή 
ενσωμάτωση επιτρέπουν την αυτοματοποίηση της ροής εργασιών. 
Επιπλέον, διευκολύνουν την άμεση ενημέρωση της ομάδας για αλλαγές 
στον κώδικα, ενισχύοντας τη συνεργασία και τη διαφάνεια μέσω της 
χρήσης του \en{GitHub}

Κατά την υλοποίηση και δοκιμή της εφαρμογής χρησιμοποιήθηκαν εργαλεία όπως το \en{Docker}, 
που εξασφαλίζει φορητότητα και σταθερότητα μέσω της ανάπτυξης και διαχείρισης \en{containers}. 
Παράλληλα η εισαγωγή του κυβερνήτη ως τεχνολογία διαχείρισης κοντέινερς αξιοποιήθηκε 
για να επιδείξει πώς μια εφαρμογή μπορεί να λειτουργεί αποτελεσματικά σε ένα τέτοιο περιβάλλον, 
αναδεικνύοντας την επεκτασιμότητα και την ευελιξία της σε πραγματικές συνθήκες

Η εφαρμογή δοκιμάστηκε σε προσομοιωμένο περιβάλλον \en{GNS3}, 
όπου αναπαραστάθηκαν διαφορετικές συνθήκες δικτύου για την 
αξιολόγηση της λειτουργικότητας και της απόδοσής της.

Επιπλέον, αξιοποιήθηκαν τεχνολογίες όπως τα \en{RESTful APIs} 
και το πρωτόκολλο \en{SSH} για τη διασύνδεση με δικτυακές συσκευές, 
ενώ η αρχιτεκτονική της εφαρμογής βασίστηκε σε μικροϋπηρεσίες 
για να διασφαλιστεί η ευελιξία και η επεκτασιμότητα. 

Η ανάπτυξη περιλάμβανε τη δημιουργία ενός φιλικού προς τον χρήστη 
περιβάλλοντος για την εισαγωγή και επεξεργασία ρυθμίσεων, καθώς και 
την ενσωμάτωση εργαλείων για την αυτοματοποιημένη εκτέλεση εντολών. 
Ιδιαίτερη έμφαση δόθηκε στις τεχνολογίες \en{containerization}, 
οι οποίες επιτρέπουν την εύκολη ανάπτυξη της εφαρμογής σε διαφορετικά περιβάλλοντα.

Συνολικά, η εργασία συνδυάζει πρακτικές αυτοματοποίησης και \en{DevOps} 
με τη χρήση σύγχρονων τεχνολογικών εργαλείων, δημιουργώντας ένα 
ολοκληρωμένο πλαίσιο διαχείρισης δικτυακών συσκευών. 
Η εφαρμογή φιλοδοξεί να αποτελέσει ένα πολύτιμο εργαλείο 
για προγραμματιστές και διαχειριστές δικτύων, συμβάλλοντας 
σημαντικά στον τομέα της αυτοματοποίησης και της παραμετροποίησης 
δικτύων.